%% sourcemem-online.tex
%%
%% A manuscript for the online sequential--simultaneous version of the
%% source memory experiment.

%% sdl: I've put a lot of extra comments in here to show you what each
%% command is doing. You can feel free to remove the comments you
%% don't need any more.

%% The `documentclass` command sets up the major look and feel of the
%% document, as well as bringing in relevant commands (sections,
%% floats, etc.) for operating within that document.
%%
%% The `apa6` document class is, as you may have expected, an
%% implementation of the publication guidelines of the sixth edition
%% of the American Psychological Association. (There is a package
%% targetting the seventh edition of the publication manual, `apa7`,
%% but it is not widely available and widely tested yet.) The
%% arguments for this package are somewhat straightforward: `10pt`
%% specifies that the base font size should be 10 points; `a4paper`
%% sets the page geometry to follow an A4 size (the default is letter
%% sized paper); `man` sets the document class to output in the
%% "manuscript" style (rather than the `jou` setting, which outputs a
%% mockup of the manuscript as it would appear as a journal article).
%% This package has a fairly good set of documentation, which you
%% should consult if you need to change things, or want to explore the
%% different options.
\documentclass[10pt, a4paper, man, biblatex]{apa6} 

%% Below are the `usepackage` commands: these import packages of
%% TeX/LaTeX functionality.
\usepackage[utf8]{inputenc} % Set the input encoding (the text of the
                            % source file) to Unicode (UTF-8).

%% Add "bibliographic resources" (i.e., BibLaTeX files).
\addbibresource{sourcemem.bib}

%% This set of commands allows plain text files to be included
%% directly in the TeX source as if it was TeX source (TeX/LaTeX can
%% be a little funny when included text as code, this bypasses that).
%% This means that tables and figures can be set up in their own
%% individual files and included in the manuscript rather than
%% copy--pasting the relevant code into manuscript document itself and
%% risking version issues.
\makeatletter % << This means that the "@" symbol will be interpreted
              %    for the following lines as being part of the command,
              %    which is usually turned off by default (to prevent
              %    people from using "system" commands.
\newcommand\primitiveinput[1] % << Define a new command.
{\@@input #1 }                % << The command body.
\makeatother                  % << Return the "@" to normal.

%% These commands are useful for drafting: they allow different
%% authors to put comments in the document at different places which
%% appear in the PDF output and (more importantly) can be searched for
%% in the LaTeX source. When compiling the final document, either
%% remove the body of the command to suppress the output on the PDF
%% or, better, fix and remove the comments.
\newcommand\jz[1]{\texttt{\small{}JZ:\@#1}}
\newcommand\afo[1]{\texttt{\small{}AFO:\@#1}}
\newcommand\sdl[1]{\texttt{\small{}SDL:\@#1}}
\newcommand\pls[1]{\texttt{\small{}PLS:\@#1}}
\newcommand\todo[1]{\texttt{\small{}TODO:\@#1}}
\newcommand\fixme[1]{\texttt{\small{}FIXME:\@#1}}

%% The bibliographic information for the manuscript. Most of this is
%% straightforward: the `apa6` document class requires authors to be
%% individually enumerated, and provides commands for different sized
%% author lists.
\title{Source memory online}
\shorttitle{Source memory online}
\fourauthors{Jason Zhou}{Author B}{Author C}{Author D}
\fouraffiliations{Melbourne School of Psychological Sciences,
  The University of Melbourne, Parkville, Victoria,
  Australia}{Melbourne School of Psychological Sciences,
  The University of Melbourne, Parkville, Victoria,
  Australia}{Melbourne School of Psychological Sciences,
  The University of Melbourne, Parkville, Victoria,
  Australia}{Melbourne School of Psychological Sciences,
  The University of Melbourne, Parkville, Victoria,
  Australia}
\leftheader{Zhou, B, C, \& D}
\authornote{Contact information and corresponding author here.}
\keywords{source memory}

\abstract{%
  This is where the abstract text will go.
}

%%%%%%
\begin{document}
\maketitle % << This is required to actually lay out the bibliographic
           %    information on the page.

Introduction here. Here is an example of a parenthetical citation
\parencite{Harlow2013} and here is an example of an in-text citation:
\textcite{Smith2016}.

\section{Method}
\subsection{Stimuli and apparatus}
Stimuli were presented on participants' personal computer monitors. Software written in Javascript using jsPsych \parencite{deLeeuw2015} controlled stimulus presentation and recorded responses. Stimuli consisted of words generated from the SUBTLEXus database, filtered for words with a length of four letters, and with frequency ratings between one and five. Words were displayed in size 24 point ``Courier New'' white font positioned in the center of a uniform mean luminance field. 

\subsection{Participants}
XX participants were recruited online through the University of Melbourne undergraduate research experience program and XX participants were recruited via Prolific. Each participant was expected to complete XX 60-minute sessions. At the end of the session, undergraduate students were granted credit towards course requirements, and paid participants were paid \$12~AUD. All participants were provided with plain language statements and consent forms, and gave informed consent prior to data collection.

\subsection{Procedure}
Participants completed the experimental tasks over XX sessions, Each of the XX sessions consisted of XX trials, which was broken up into XX blocks of XX items each. Blocks were comprised of a study phase, a math distractor phase, a recognition phase, and finally a source recall phase. There were two conditions in this experiment, a simultaneous encoding condition and a sequential encoding condition, with all other phases being identical between the conditions. 

In the sequential encoding condition, participants were presented with a black marker positioned on a randomly generated angle on the outline of a circle at the start of each trial for 600~ms. The presentation of the marker was followed by the display of a word in the center of the screen for 1,500~ms. To ensure that participants attended to the source information, they were instructed to indicate the previous location of the cross on the blank target circle using a computer mouse. Responses made within $\pi/4$~radians of the true target location were classified as attended and advanced participants to the next item. Responses further away were deemed unattended and the words ``TOO DISTANT'' was displayed for 1,000~ms, then the location was then re-presented and the verification task was repeated.

In the simultaneous encoding condition, participants were presented with the marker and the word simultaneously for 1,000~ms. Instead of being positioning the word in the centre of the screen, in the simultaneous encoding condition, the word was positioned at the same angle as the marker, offset by a longer radius. The location of the word relative to the marker was determined by the sector the angle was in, with the word being offset to one of eight points on the bounds of the text box, corresponding to the middle of each of the four sides, and the four corners (i.e. in the North sector, the anchor was the bottom middle of the text box, while in the Northeast sector the anchor was the bottom left of the text box). As with the sequential condition, a verification task followed each presentation, which was repeated until participants reproduced the location to within XX radians of the presented angle.

After studying each of the items for that block, participants were then instructed to complete a distractor task, which involved 30~seconds of arithmetic problems. These problems were presented as three single digit integers, which summed to a fourth number which would either be the correct sum, or a number that was one higher or lower than the actual sum. Participants would indicate if the sum was correct by pressing the keys 0 (false) or 1 (true).

In the recognition phase, participants were shown a shuffled list of 10 previously studied items and 10 foils and asked to rate each item on a six-point Old/New confidence scale. Participants responded by pressing a number from 1 to 6 on their keyboard, with 1 representing ``Sure New'' and 6 representing ``Sure Old''.

Finally, in the source memory retrieval task, participants were cued with the words for 1,500~ms, and then indicated the recalled location by a moving the mouse from the starting point in the centre of the circle to a point on the circumference of the response circle. There was no time limit on the decision task.

\section{Results}
Results go here

\section{Discussion}
Discussion goes here

\subsection{Conclusion}
Conclusion goes here


\printbibliography
\end{document}

%%  VVVV This is a bunch of stuff to configure emacs.
%%% Local Variables:
%%% mode: latex
%%% coding: utf-8
%%% ispell-local-dictionary: "en_US"
%%% End:
% LocalWords:  VSTM psychophysical 
